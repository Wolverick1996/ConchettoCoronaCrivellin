%\documentclass[11pt]{article}
%\usepackage{fullpage}

%\begin{document}

\subsection{Selected Architectural Styles and Patterns}
\subsubsection{Architectural Styles}
\hspace{-\parindent}\textbf{Client and Server}\\

The main architectural style adopted for TrackMe system is the client-server one, the most well known and used architectural style for distributed applications. It will be adopted in the 3-tier variant, with the presentation layer on the client (the mobile app), the application layer on the application server and the data layer on the database server.

The main advantages of this choice are the clear decoupling between data and logic, the possibility to increase the portability reaching clients in the most easy way and the availability of a lot of COTS components to  develop the system in a very cost-effective way.\\

\hspace{-\parindent}\textbf{Thin Client}\\

The thin client approach has been followed while designing the interaction among users’ machines and the system itself. All the main logic is implemented by the Application Server that has a sufficient computing power and can manage concurrency issues in an efficient way. On the other hand, the mobile application are in charge of presentation only and they do not involve decision logic.

\subsubsection{Design Patterns}
\textbf{Model-View-Controller}\\

The mobile (and wearable) applications follow the MVC (Model-View-Controller) software design pattern, that better supports the client-server architecture, because it closely follows the division between data (Model), logic (Controller) and presentation (View) present also in our 3-tier architecture.\\
\\
\textbf{Observer}\\

The observer or publish-subscribe pattern will also be used, allowing the various components of the system to register themselves and react to event raised by other components. This will be useful in implementing the EmergencyManager that always has to be notified in order to compare data and thresholds.
\\

%\end{document}