%\documentclass[11pt]{article}
%\usepackage{fullpage}
%\usepackage{graphicx}

%\begin{document}

\subsection{Purpose}
This document is intended to give a broad description of the three services offered by TrackMe: Data4Help, AutomatedSOS and Track4Run. This will be done by a detailed presentation of the proposed solution and its purpose, listing its objectives, and the requirements and assumptions through which they will be achieved. The document is meant to be used by the clients, users and also by the parties designated with the task of creating the specified system, mainly the system and requirements analysts, the project managers, software developers and testers.\newline
\\
The company TrackMe wants to offer third parties the ability to monitor the position and health status of groups of people. To do this they decided to create an app, called TrackMe, which can offer this type of service and at the same time also allows ordinary users to register and make their data available in exchange for continuous monitoring. Data4Help is a service that manages the data collected by users, who can decide whether to be available to be completely traceable or to participate in anonymous groups from which third parties can draw information. The system uses the smartphone GPS, while the acquisition of health data relies on sensors in smartwatch and smart band. TrackMe also offers free the possibility to activate the AutomatedSOS service, recommended for elderly people but open to anyone, thanks to which the monitoring of the health status of registered users will be taken to a subsequent level: in fact, if the vital parameters (heartbeat and blood pressure) exceed certain thresholds, an emergency call will be automatically activated to mobilize an ambulance, with the guarantee that this call will take place within 5 seconds from the time the thresholds are exceeded. Finally, TrackMe offers an additional service to take advantage of the data collected with Data4Help: Track4Run. This last service, on payment, if activated allows you to be tracked as an athlete participating in a race, therefore allows the organizers to create events that other users can decide whether to participate or simply follow the race from their smartphone.

\subsection{Scope}
\subsection{Definitions, Acronyms, Abbreviations}
%\end{document}
