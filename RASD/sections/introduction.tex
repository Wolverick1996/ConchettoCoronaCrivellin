%\documentclass[11pt]{article}
%\usepackage{fullpage}
%\usepackage{graphicx}

%\begin{document}

\subsection{Purpose}
This document is intended to give a broad description of the three services offered by TrackMe: Data4Help, AutomatedSOS and Track4Run. This will be done by a detailed presentation of the proposed solution and its purpose, listing its objectives, and the requirements and assumptions through which they will be achieved. The document is meant to be used by the clients, users and also by the parties designated with the task of creating the specified system, mainly the system and requirements analysts, the project managers, software developers and testers.\newline

The TrackMe company wants to offer third parties the ability to monitor the position and health status of groups of people. To do this they decided to create an app, called TrackMe, which can offer this type of service and at the same time also allows ordinary users to register and make their data available in exchange for continuous monitoring.

Data4Help is a service that manages the data collected by users, who can decide whether to be available to be completely traceable or to participate in anonymous groups from which third parties can draw information. The system uses the smartphone GPS, while the acquisition of health data relies on sensors in smartwatch and smart band.

TrackMe also offers free the possibility to activate the AutomatedSOS service, recommended for elderly people but open to anyone, thanks to which the monitoring of the health status of registered users will be taken to a subsequent level: in fact, if the vital parameters exceed certain thresholds, an emergency call will be automatically activated to mobilize an ambulance, with the guarantee that this call will take place within 5 seconds from the time the thresholds are exceeded.

Finally, TrackMe offers an additional service to take advantage of the data collected with Data4Help: Track4Run. This last service allows people to be tracked as athletes participating in a race, therefore allows the organizers to create events that other users can decide whether to participate or simply follow the race from their smartphone.

\subsection{Scope}
\subsubsection{Description of the given problem}
As already explained above, TrackMe is a versatile app that once purchased provides three good services. Being an app for both companies that want to access user data, and for common users who want to monitor their data, there will be two slightly different interfaces. The generic user will be able to view his / her collected data, accept (or reject) requests to use his / her data by third parties and access the Track4Run section where he / she can create, participate or simply follow sports events. On the other hand, a user with access to a company account will have much more limited functions since the home page will be able to access lists of users who are registered for data collection, or a search section from which they will be able to search for new users or anonymous groups to sign up for. When a company account wishes to subscribe to the data of a specific user, a notification will be sent to the user in question, this will accept or reject the request, then the company can cancel the registration as the user who provides the data will prevent access to your data.

Regarding the Track4Run section, users will be able to create public or private competitions and to do so it will be necessary to insert a valid race path, consisting of: a start, an arrival (also coinciding in case of at least one further stage) start date and time ( that can be postponed by the organizer or by other users who have been defined as co-organizers), in case you want to create a tender invitation, the list of invited users and finally the registration fee (optional). If you want to participate or follow a race, you can do so by selecting one of the races in the NEARBY, YOUR RACES sections.

The AutomatedSOS service is activated automatically if the user is over the age of 60 at the time of enrollment, otherwise it can be activated from the settings and always works in the background, checking that the health data does not exceed certain thresholds.
\subsubsection{Goals}
[G1] - Users and third parties can be recognized by providing a form with their data\newline

\hspace{-\parindent}[G2] - Allow third parties to access to the data of some specific individuals\newline

\hspace{-\parindent}[G3] - Allow third parties to access to anonymized data of groups of individuals\newline

\hspace{-\parindent}[G4] - Allow third parties to subscribe to new data and to receive them as soon as they are produced\newline

\hspace{-\parindent}[G5] - Allow the users, through the AutomatedSOS service, to come help by an ambulance when such parameters fall below certain thresholds\newline

\hspace{-\parindent}[G6] - Allow racing organizers, through the Track4Run service, to manage runs and define a path for runs\newline

\hspace{-\parindent}[G7] - Allow racing participants to enroll runs\newline

\hspace{-\parindent}[G8] - Allow racing spectators to see on a map the position of all runners during the run\newline

\hspace{-\parindent}[G9] - Allow users to monitor their own health status

\subsection{Definitions, Acronyms, Abbreviations}
\subsubsection{Definitions}
\begin{itemize}
\item \textbf{Generic user}: a user who does not have special accordances with TrackMe and who is the object of third parties surveys.
\item \textbf{Third party}: user who has an accordance with TrackMe; the latter recognizes its profile as a public entity that can carry out surveys.
\item \textbf{Data (temporary and permanent)}: parmeters recordered over time (temporary, that change from time to time) or once for all (permanent, that don't change).
\item \textbf{Survey}: a research carried out on some specific parameters of one or more individuals over time.
\item \textbf{By invitation run}: a run whose partecipants are only the invitated ones.
\item \textbf{Open run}: a public run, everyone (according to the it's rules) can join it.
\item \textbf{Identifiable individual}: a generic user is defined so (with a reasonable approximation) if his/her complete name or fiscal code is known or if he/she forms part of a group of fewer than 1000 people that is object of a survey.
\end{itemize}
\subsubsection{Acronyms}
\begin{itemize}
\item RASD – Requirement Analysis and Specification Document
\item API - Application Programming Interface
\item GPS - Global Positioning System
\item BPM - Beats (of heart) Per Minute
\item NUE - Numero Unico Per Emergenze
\end{itemize}
\subsubsection{Abbreviations}
\begin{itemize}
\item \verb|[|Gn\verb|]|: n-th goal
\item \verb|[|Dn\verb|]|: n-th domain assumption
\item \verb|[|Rn\verb|]|: n-th functional requirement
\end{itemize}

\subsection{Document Structure}
Chapter 1 gives an introduction to the problem and describes the purpose of the TrackMe application. The scope of the application is defined by stating the goals and the description of the problem.\newline

Chapter 2 presents the overall description of the project. The product perspective includes details on
the shared phenomena and the domain models. The class diagram describes the domain model used,
and the state diagram analyzes the process of arranging a meeting and reaching it in time. Here the
majority of functions of the system are more precisely specified, with respect to the already mentioned
goals of the system. In the user characteristics section the types of actors that can use the application
are described.\newline

Chapter 3 contains the external interface requirements, including: user interfaces, hardware interfaces,
software interfaces and communication interfaces. Few scenarios describing specific situations are listed
here. Furthermore, the functional requirements are defined by using use case and sequence diagram.
The non-functional requirements are defined through performance requirements, design constraints
and software system attributes.\newline

Chapter 4 includes the Alloy model with its representation and the discussion of its purpose.\newline

Chapter 5 shows the effort spent by each group member while working on this project.\newline

Chapter 6 includes the reference documents.

%\end{document}